\documentclass[fontset=none]{ctexart}
\ctexset{fontset=windows}   %指定字体库为windows
\usepackage{fancyhdr}
\setlength{\headheight}{13pt}  % 调整页眉高度
\usepackage{float}
\usepackage{amsmath,amsthm,amssymb,amsfonts}
\usepackage[left=3cm,right=3cm,top=2.5cm,bottom=2.5cm]{geometry}
\usepackage{graphicx}
\usepackage{epstopdf}
\usepackage{subfigure}
\pagestyle{fancy}
\fancyhf{} % 清空默认页眉页脚
\fancyhead[L]{QQ:17950026}
\fancyhead[R]{作者:姜鱼}
\fancyhead[C]{兼听则明,偏听则暗}

\usepackage{lmodern}
\makeatletter
\renewcommand{\maketitle}{
  \begin{flushleft} % 整体左对齐
    {\huge\bfseries \@title \par} % 标题
    \vspace{0.5em}
    {\large \bfseries \textit{作者:} \@author \par} % 作者(添加自定义文字)
    \vspace{0.5em}
    {\bfseries 发布日期:\@date} % 日期(调整格式)
  \end{flushleft}
  \thispagestyle{empty} % 可选:隐藏页眉页脚
  \vspace{1cm} % 标题块后间距
}
\makeatother

\begin{document}

    \title{一种用分母结构转化分子的思路}
    \author{姜鱼}
    \date{\today}
    \maketitle
    \songti
    
    注意到一个等式变换:$$\frac{x}{x+3}=1-\frac{3}{x+3}$$
    
    原来的函数中,分子和分母处各有一个可变量$x$,通常处理这种函数的时候,这会使分析更加复杂。我们只希望变量出现在分母处,而使分子变为更加简单的常数$3$,这样的结构变换大大地简化了我们的分析难度。

    仔细看变换的本质原理,关键在这个“1”处。我们将这个1视为分式$\frac{1}{1}$,继而让分子分母同时乘上$x+3$,这样就可以使分母的结构转移到分子上:
    
    $$
    1-\frac{3}{x+3}=\frac{x+3}{x+3}-\frac{3}{x+3}
    $$

    分母的部分分为“$x$”和“3”,我们要保留$x$而去掉3的部分,于是有了这个等式。生动形象的来说,我们的分母有$x+3$,而分子从$x$置换成了3。显然如果分子是3,我们也可以如此一般置换为$x$。

    我们将其比喻为一个「置换反应」。

    接下来分母更加复杂,但是我们依然可以利用类似的思路去置换我们想要的部分。例如:$$\frac{x}{\left( x+3 \right) \left( x+4 \right)}=\frac{1}{x+4}-\frac{3}{\left( x+3 \right) \left( x+4 \right)}=\frac{x+3}{\left( x+4 \right) \left( x+3 \right)}-\frac{3}{\left( x+3 \right) \left( x+4 \right)}$$

    这个时候我们会发现,需要分式上下同乘$x+3$的部分不再是1,而是分式$\frac{1}{x+4}$,结构或许变得复杂了,但是思路其实还是如法炮制。依然是利用分母中同时包含变量x和常数的特性,将分子的变量成分置换为常数。当然,对象有所转变,形式有所复杂,但是手段是不变的。甚至如果我们的目标是置换另一个变量y,也可以实现。例如:$$\frac{x}{\left( x+y \right) \left( x+y+1 \right)}=\frac{1}{x+y+1}-\frac{y}{\left( x+y \right) \left( x+y+1 \right)}$$或者$$\frac{x}{\left( x+y \right) \left( x+y+1 \right)}=\frac{1}{x+y}-\frac{y+1}{\left( x+y \right) \left( x+y+1 \right)}$$思路上是一致的。

    当然,其实有更简单的手法,就是把复杂因式变成熟悉的样子,比如这样看待:

    $$
    \frac{x}{\left( x+k \right) \left( x+k+1 \right)}=\frac{x}{\left( x+k \right)}\cdot \frac{1}{\left( x+k+1 \right)}=\left[ 1-\frac{k}{x+k} \right] \cdot \frac{1}{\left( x+k+1 \right)}=\frac{1}{\left( x+k+1 \right)}-\frac{k}{\left( x+k \right) \left( x+k+1 \right)}
    $$

    从分母中置换分子的思路可以帮我们处理很多麻烦。在需要的时候把当前不想要的分子结构,置换为分母中包含的其他我们更愿意见到的结构,可以使我们的解题思路更开阔。








\end{document}