\documentclass[fontset=none]{ctexart}
\ctexset{fontset=windows}   %指定字体库为windows
\usepackage{fancyhdr}
\setlength{\headheight}{13pt}  % 调整页眉高度
\usepackage{float}
\usepackage{amsmath,amsthm,amssymb,amsfonts}
\usepackage[left=3cm,right=3cm,top=2.5cm,bottom=2.5cm]{geometry}
\usepackage{graphicx}
\usepackage{epstopdf}
\usepackage{subfigure}
\pagestyle{fancy}
\fancyhf{} % 清空默认页眉页脚
\fancyhead[L]{QQ:17950026}
\fancyhead[R]{作者:姜鱼}
\fancyhead[C]{兼听则明,偏听则暗}

\usepackage{lmodern}
\makeatletter
\renewcommand{\maketitle}{
  \begin{flushleft} % 整体左对齐
    {\huge\bfseries \@title \par} % 标题
    \vspace{0.5em}
    {\large \bfseries \textit{作者:} \@author \par} % 作者(添加自定义文字)
    \vspace{0.5em}
    {\large\bfseries \textit{日期:}\@date}% 日期(调整格式)
  \end{flushleft}
  \thispagestyle{empty} % 可选:隐藏页眉页脚
  \vspace{1cm} % 标题块后间距
}
\makeatother

\begin{document}
\title{由鸡兔同笼问题从方程组步入矩阵}
\author{姜鱼}
\date{\today}
\maketitle
\renewcommand{\thesubsection}{\arabic{subsection}} % 仅显示subsection自身的数字
\songti

考虑这么一个二元一次方程组:$$
\begin{cases}
	3x+7y=114\\
	5x-y=514\\
\end{cases}
$$  

我们考虑的不是如何对它求解,而是我们如何写下的它。

\subsection{譬如一个鸡兔同笼问题}
现有10个头,28个脚,我们写下的方程应该是:$$\begin{cases}
	x+y=10\\
	2x+4y=28\\
\end{cases}$$

我们向来写方程,都是一行一行写。意思就是说,我们先设鸡有x个,兔有y个。第一行的思路是:我们总共有10个动物,就有10个头,其中鸡的个数对总头的贡献为1,兔头对总头贡献为1,于是写出了$1x+1y=10$。第二行的思路是:我们总共有28个脚,其中鸡对脚数的贡献是2,所以2作为系数乘在x边,而兔对脚数贡献为4,所以4作用于y,由此写出$2x+4y=28$。

于是分成两行,写了两个方程\begin{equation*}
  \left\{
  \begin{aligned}
      &\ \ x + y = 10 \quad && \text{……头方程} \\
      &\ \ 2x + 4y = 28 \quad && \text{……脚方程}
  \end{aligned}
  \right.
\end{equation*}

一行方程表示一个事件,可以看出,头和脚是两个完全不同维度的事物,我们以事件为线索,分别对两个维度的事件写等式,串联起两个独立维度的方程,便可以充分地求解出一个方程。但是一定要确保两个方程由两个完全独立的事件所给出,这样我们才能确保信息是有效的,如果我们写出这样的两个方程:$\begin{cases}
	x+y=10\\
	2x+2y=20\\
\end{cases}$,虽然看似是联立两个二元一次方程,实际上两个方程根据一个事件所给出,是并不独立的两个维度,则不能对其有效求解。

为什么我们要刻意强调以事件为线索地列方程呢?因为事实上,真正的鸡兔同笼问题的求解思路,和我们这样列二元一次方程组的思路大相径庭。

我们这次区分的事件并非“头”和“脚”,而是真正从“鸡”和“兔”出发,字面意义地去思考“鸡兔”同笼的问题。假设笼中全是鸡,那么10头鸡总共有$10\times 2=20$只脚,然而总共有28个脚,多出来的8个脚由兔子提供,每个鸡换成兔子后,都会使脚多2只,因此多出的8只脚来自$8\div2=4 $只兔。这是传统的求解鸡兔同笼的思路,仔细审视其思路,其实依然并非完全不能用方程组表示,只是要适当处理一下:\begin{equation*}
  \left\{
  \begin{aligned}
      &\ \ x = 10-y \quad && \text{\lishu …… 先带入$x=10$,再去掉其中y的部分} \\
      &\ \ 2x = 2×(10-y)-(2y-8) \quad && \text{\lishu …… 由此导出$2y-8=0$,算出兔数}
  \end{aligned}
  \right.
\end{equation*}

这次我们把鸡看做整体线索,同时考察“头系数=1”和“脚系数=2”对鸡数的作用效果,最终通过鸡兔自身性质(脚的数量差2)的比较,得出鸡和兔的数量。我们的计算离不开10和28两个结果导向的媒介,但是我们解析的方式截然不同。1头2脚整体算作属于“鸡”的性质,在这里共同作用于鸡数$x$的时候,再用方程组分成两行来表述,就违和了许多。因为我们的思路其实并没有如最初写方程组那般分成“两个维度”去思考和分析问题。

向量方程就是这样诞生的。

如果作用于$x$的整体系数作为鸡的性质,从而整体看待,用另一种方式记为$\left[ \begin{array}{c}
	1\\
	2\\
\end{array} \right]×x$,上面的数对应方程组中上面那行,下面的数对应方程组中下面那行。我们继续用这种思路把整个方程组写为:$$\left[ \begin{array}{c}
	1\\
	2\\
\end{array} \right] x=\left[ \begin{array}{c}
	1\\
	2\\
\end{array} \right] \left( 10-y \right) -\left[ \begin{array}{c}
	0\\
	2y-8\\
\end{array} \right] $$

进而回头再利用$(10-y)=x$导出$\left[ \begin{array}{c}
	0\\
	2y-8\\
\end{array} \right] =0$ 或 $2y-8=0$,最终解出y再进一步解出x来实现求解鸡兔同笼。

我们写出$\left[ \begin{array}{c}
	1\\
	2\\
\end{array} \right] x$的结构并非难点,但是为了构造等式后面那一长串的内容可谓煞费苦心。我们虽然知晓二元一次方程组的解法,而仍愿意探究鸡兔同笼的原理,其在于原则上有趣的思路多多益善,而并不为其复杂的形式所遏制。但是一味地奇技淫巧也并非我们追逐的目标,适当的去芜存菁才是学习数学的本质。

我们的思路是以完全不同的结构$\left[ \begin{array}{c}
	1\\
	2\\
\end{array} \right] x$代替原先两行方程组的结构,进一步说,是用这种别样的思路来表述变量之间的关系。其本质只是思路上从以\textbf{分解结果计算头和脚为核心线索}改为了\textbf{以变量元为核心线索}进行方程的书写,所以其实繁杂的形式并不是必要的,我们进行了一系列的构造,其最终的无非还得从10只头,28个脚入手。对于y,也无需将其进行复杂的构造,等同于x一般去书写对其作用的,代表其性质的系数组$\left[ \begin{array}{c}
	1\\
	4\\
\end{array} \right] y$。所以我们进一步借鉴二元一次方程组的简洁形式,将其以新的书写思路改造为:
$$
\left[ \begin{array}{c}
	1\\
	2\\
\end{array} \right] x+\left[ \begin{array}{c}
	1\\
	4\\
\end{array} \right] y=\left[ \begin{array}{c}
	10\\
	28\\
\end{array} \right] 
$$

这样的结构简直焕然一新!

\subsection{从新的角度重新解构线性方程组}
对于任意的方程,例如那个$
\begin{cases}
	3x+7y=114\\
	5x-y=514\\
\end{cases}
$我们都可以将其改写为新的结构:$\left[ \begin{array}{c}
	3\\
	5\\
\end{array} \right] x+\left[ \begin{array}{c}
	7\\
	-1\\
\end{array} \right] y=\left[ \begin{array}{c}
	114\\
	514\\
\end{array} \right] $。
这样做并非我想自讨苦吃地专注于形式和结构的改造,而是以全新的思路去处理变量关系。提炼结构,是为了更好地挖掘其中的性质。最终即便我们任意写的方程中,每一行没有了一个实际场景的意义作为支撑,例如$x,y$未必再代表鸡和兔,数表$\left[ \begin{array}{c}
	3\\
	5\\
\end{array} \right] $也不再代表鸡所具有的头和脚的数量性质。但是仍然有系数作用于变量的方程,我们都可以挖掘出其本质的共性,故鸡兔同笼的问题最终可以总结为:可以将方程组分离成变量与系数的两个部分,每一个变量元有一个对应的系数列表。

对于多元方程组$\begin{cases}
	a_1x_1+\cdots +d_1x_n\\
	\,\,\vdots \,\,   \ddots \,\,  \vdots\\
	a_nx_1+\cdots +d_nx_n\\
\end{cases}=\begin{array}{c}
	b_1\\
	\vdots\\
	b_n\\
\end{array}$,我们都可以将其转化为$\left[ \begin{array}{l}
	a_1\\
	\vdots\\
	a_n\\
\end{array} \right] x_1+\cdots +\left[ \begin{array}{l}
	d_1\\
	\vdots\\
	d_n\\
\end{array} \right] x_n=\left[ \begin{array}{c}
	b_1\\
	\vdots\\
	b_n\\
\end{array} \right] $的形式,前者以每行为导向分析问题,解构结果条件。后者以变量元一列为导向按列书写方程组,侧重于表达与变量元有关的性质。

我们把横行的数量称为“维度”,体现的是事件,按维度书写方程,是以事件为线索集合不同变量的数据。我们把纵列的数量称为“自由度”,代表的是有多少未知的自由变量,按自由度书写方程,是以变量为线索集合不同维度事件中所有的性质的数据。

\subsection{方程组与向量}
进一步地分析结构,我们依旧可以再次给出方程组一个新的理解角度。中学阶段向量的形式写作$\vec{a}=\left( a_1,a_2,\cdots ,a_n \right) $,本质是作为一群数据的集合体。同样是作为一群数据的集合体,我们可以把形式为$\vec{a}=\left( a_1,a_2,\cdots ,a_n \right) $的向量改写为$\vec{a}=\left[ \begin{array}{c}
	a_1\\
	\vdots\\
	a_n\\
\end{array} \right] $,也可以把形式为$\vec{a}=\left[ \begin{array}{c}
	a_1\\
	\vdots\\
	a_n\\
\end{array} \right] $的向量改写为$\vec{a}=\left( a_1,a_2,\cdots ,a_n \right) $。

于是之前我们竖向写的数表,本质和向量无异,则意味着可以当做向量处理(这也是为什么把横行的数量成为维度)。那么接下来有趣的就来了。首先,我们把$$\left[ \begin{array}{l}
	a_1\\
	\vdots\\
	a_n\\
\end{array} \right] x_1+\cdots +\left[ \begin{array}{l}
	d_1\\
	\vdots\\
	d_n\\
\end{array} \right] x_n=\left[ \begin{array}{c}
	b_1\\
	\vdots\\
	b_n\\
\end{array} \right] $$的模样变为$$x_1\left[ \begin{array}{l}
	a_1\\
	\vdots\\
	a_n\\
\end{array} \right] +\cdots +x_n\left[ \begin{array}{l}
	d_1\\
	\vdots\\
	d_n\\
\end{array} \right] =\left[ \begin{array}{c}
	b_1\\
	\vdots\\
	b_n\\
\end{array} \right] $$相当于之前把数表的一群东西当做了$x_n$的系数,如今位置互换,把$x_n$当成了一个个向量的系数。

总的头和脚的数量,分别由鸡头鸡脚的数量,和兔头兔脚的数量构成,而描述一只鸡的特征的向量$\boldsymbol{A}$,里面的两个维度分别包含了两个独立数据头和脚:$\left[ \begin{array}{c}
	1\\
	2\\
\end{array} \right]$,那么鸡的数量则为其加权,去构成整体的头脚数$\left[ \begin{array}{c}
	1x\\
	2x\\
\end{array} \right]$。同样的,我们用兔数为兔的特征加权得到$\left[ \begin{array}{c}
	1x\\
	4x\\
\end{array} \right]$,并将鸡兔加权后的头脚数据相加,就可以构成整体的头脚数据$\left[ \begin{array}{c}
	10\\
	28\\
\end{array} \right] 
$。

广而推之,将每个变量$x_n$在不同维度的系数数据指标都整合为一个整体向量,令$\boldsymbol{A}=\left[ \begin{array}{l}
	a_1\\
	\vdots\\
	a_n\\
\end{array} \right] $,再试变量为赋向量的权。将权$x_1,x_2,...,x_n$如法炮制地赋予向量$\boldsymbol{A}_1,\boldsymbol{A}_2,...,\boldsymbol{A}_n$,我们的线性方程组则可以直接改写为
$$\boldsymbol{A}_1x_1+\cdots +\boldsymbol{A}_nx_n=\boldsymbol{b}$$

不断地精炼,浓缩结构,我们把一个线性方程组浓缩为了单行的一个方程。其中多行多维度的性质涵盖在了向量之内。接下来我们继续提炼结构,之前将向量作为$x_n$的系数,而今,我们也可以把将多个自由变量$x_n$视为系数向量的加权,认真思考如下这句话:解线性方程组,如同找到每个系数向量的权$x_1\to x_n$,令n个基向量最终指向目标向量$\boldsymbol{b}$,并作定义:
$$\boldsymbol{A}_1x_1+\cdots +\boldsymbol{A}_nx_n\Longrightarrow \left[ \begin{matrix}
	\boldsymbol{A}_1&		\boldsymbol{A}_2&		\cdots&		\boldsymbol{A}_n\\
\end{matrix} \right] \cdot \left[ \begin{array}{c}
	\boldsymbol{x}_1\\
	\boldsymbol{x}_2\\
	\vdots\\
	\boldsymbol{x}_n\\
\end{array} \right]=\boldsymbol{b} $$
其中$\left[ \begin{matrix}
	\boldsymbol{A}_1&		\boldsymbol{A}_2&		\cdots&		\boldsymbol{A}_n\\
\end{matrix} \right]$是代表不同权所具有的特征的向量,通常他们共同组成了n维空间的基底,而$\left[ \begin{array}{c}
	\boldsymbol{x}_1\\
	\boldsymbol{x}_2\\
	\vdots\\
	\boldsymbol{x}_n\\
\end{array} \right]$则为权向量,为不同的$A_1,...,A_n$加权,生成n维空间中的$\boldsymbol{b}$向量。

同时由于$\boldsymbol{A}_2=\left[ \begin{array}{c}
	{\boldsymbol{a}_1}_2\\
	{\boldsymbol{a}_2}_2\\
	\vdots\\
	{\boldsymbol{a}_n}_2\\
\end{array} \right] $,所以将其每一位展开,$\left[ \begin{matrix}
	\boldsymbol{A}_1&		\boldsymbol{A}_2&		\cdots&		\boldsymbol{A}_n\\
\end{matrix} \right] \cdot \left[ \begin{array}{c}
	\boldsymbol{x}_1\\
	\boldsymbol{x}_2\\
	\vdots\\
	\boldsymbol{x}_n\\
\end{array} \right] $中的基向量部分,也可以表述为$$\left[ \begin{matrix}
	\boldsymbol{A}_1&		\begin{array}{c}
	{\boldsymbol{a}_1}_2\\
	{\boldsymbol{a}_2}_2\\
	\vdots\\
	{\boldsymbol{a}_n}_2\\
\end{array}&		\boldsymbol{A}_3\ \ \,\cdots&		\boldsymbol{A}_n\\
\end{matrix} \right]$$或者全部展开,形成更大的数表$$\left[ \begin{matrix}
	{\boldsymbol{a}_1}_1&		{\boldsymbol{a}_1}_2&		\cdots&		{\boldsymbol{a}_1}_n\\
	{\boldsymbol{a}_2}_1&		\ddots&		&		{\boldsymbol{a}_2}_n\\
	\vdots&		&		\ddots&		\vdots\\
	{\boldsymbol{a}_n}_1&		{\boldsymbol{a}_n}_2&		\cdots&		{\boldsymbol{a}_n}_n\\
\end{matrix} \right]$$这样一个包含了一个方程组全部系数的数表,被命名叫系数矩阵。

矩阵表示的是一整个线性方程组的系数列表,矩阵运算被定义的目的是为了结构化运算。如同提炼结构是为了更好地挖掘其中的性质,我们做了许多复杂的操作,实际上只是顺着一些基本的逻辑,把一块方程组拆成了三个独立的部分:

(1)系数矩阵$A=\left[ \begin{matrix}
	{\boldsymbol{a}_1}_1&		{\boldsymbol{a}_1}_2&		\cdots&		{\boldsymbol{a}_1}_n\\
	{\boldsymbol{a}_2}_1&		\ddots&		&		{\boldsymbol{a}_2}_n\\
	\vdots&		&		\ddots&		\vdots\\
	{\boldsymbol{a}_n}_1&		{\boldsymbol{a}_n}_2&		\cdots&		{\boldsymbol{a}_n}_n\\
\end{matrix} \right]$

(2)包含自由变量的权向量$\boldsymbol{x}=\left[ \begin{array}{c}
	x_1\\
	\vdots\\
	x_n\\
\end{array} \right] $

(3)一个常数向量$\boldsymbol{b}=\left[ \begin{array}{c}
	\boldsymbol{b}_1\\
	\vdots\\
	\boldsymbol{b}_n\\
\end{array} \right] $

于是我们可以将三个结构整合,把线性运算最终归纳为$$A\boldsymbol{x}=\boldsymbol{b}$$

\subsection{总结}
我们捋了一遍从线性方程组到矩阵的过程:
\begin{align*}
	\begin{cases}
		x+2y+7z=11\\
		-2x+5y+4z=45\\
		-5x+6y-3z=14\\
	\end{cases} &\Longrightarrow 
	\left[ \begin{array}{c}
		1\\
		-2\\
		-5\\
	\end{array} \right] x+\left[ \begin{array}{c}
		2\\
		5\\
		6\\
	\end{array} \right] y+\left[ \begin{array}{c}
		7\\
		4\\
		-3\\
	\end{array} \right] z=\left[ \begin{array}{c}
		11\\
		45\\
		14\\
	\end{array} \right] 
	\\&\Longrightarrow \left[ \begin{matrix}
		\vec{A_1}&		\vec{A_2}&		\vec{A_3}\\
	\end{matrix} \right] \left[ \begin{array}{c}
		x\\
		y\\
		z\\
	\end{array} \right]=\left[ \begin{array}{c}
		11\\
		45\\
		14\\
	\end{array} \right] 
	\\&\Longrightarrow \left[ \begin{matrix}
		1&		2&		7\\
		-2&		5&		4\\
		-5&		6&		-3\\
	\end{matrix} \right] \left[ \begin{array}{c}
		x\\
		y\\
		z\\
	\end{array} \right] =\left[ \begin{array}{c}
		11\\
		45\\
		14\\
	\end{array} \right] 
	\\&\Longrightarrow A\vec{x}=\vec{b}
\end{align*}\\同时,也学到了三种看待方程组运算的角度:

一种是横着看:按照独立的事件集合不同元素的等式。

一种是竖着看:按照自由变量的性质集合不同维度的系数。

一种是从向量看,把线性方程组视为若干系数向量的线性组合。



这是一篇来自姜鱼的文章,于学习时偶然间顿有灵感,于是记录自己的学习心得,整理为文章。多有疏漏,欢迎各位与我交流,批评指正,谢谢大家阅读。




\end{document}