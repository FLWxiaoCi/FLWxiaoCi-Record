\documentclass[fontset=none]{ctexart}
\ctexset{fontset=windows} %指定字体库为windows

\usepackage{fancyhdr}
\usepackage{ctex}
\usepackage{float}
\usepackage{amsmath,amsthm,amssymb,amsfonts}
\usepackage[left=3cm,right=3cm,top=2.5cm,bottom=2.5cm]{geometry}
\usepackage{graphicx}
\usepackage{epstopdf}
\usepackage{subfigure}
\pagestyle{fancy}%清除原页眉页脚样式



\begin{document}

\title{微分形式的不变性}
\author{姜鱼}
\date{\today}
\maketitle

\songti 我们接触微分,往往都是从导数出发,我们先定义了极限,再将比值的极限定义为导数$\frac{dy}{dx}$。最初,对于求导符号$\frac{dy}{dx}$来说,其中的$\frac{d}{dx}$可视作一个整体,但是随即,我们还是妥协于这种分式的形式,把分子分母独立拿开,变成单独的$dx$和$dy$。        
\begin{equation}
    dy=f'\left( x \right) dx
\end{equation}
这叫做微分,可以视为其表达着具体的含义,也可以将其仅作为符号记法。但是记号永远不是最值得关注的部分,微分的意义在于我们可以望文生义地推导,例如在一个简单的解微分方程中:
\begin{equation}
    \frac{dy}{dx}=2txy
\end{equation}
由于分式可以拆开,我们可以将$dx$和$dy$试做整体,很方便地进行移动,若使左右两侧变量统一,则有:
\begin{equation}
    \frac{dy}{y}=2txdx
\end{equation}
将其左右两侧分别积分,有:
\begin{equation}
    \ln y=tx^2+C
\end{equation}
最后得到
\begin{equation}
    y=e^{tx^2+C}=Ae^{tx^2}
\end{equation}

如果$y$和$x$满足如上的微分方程,我们就可以通过这样的过程,求得$y$和$x$的函数关系。初次见面,得益于我们用分式表示导数,并用$d$和$\int$作为互逆的记号处理无穷小量,或许这种望文生义的写法比较容易被理解和接受,那么我们思考一个更有意思的问题。

如果引入一个参数$t$作为中间变量,且$y$与$x$都有关于$t$的直接的约束$y=f(t)$和$x=g(t)$,那么此时$y$和$x$的函数关系不能用$f$来表示,而是用新的函数$F$来表示其复合函数的关系
\begin{equation}
    y=F\left( x \right) =F\left[ g\left( t \right) \right]
\end{equation}
则$y$的微分将变得有趣了起来。先从函数第一定义出发,用变量的思维审视其中的变化关系,考虑到变量$y$对不同变量$x$和$t$的求导,有$\frac{dy}{dt}=f'\left(t\right) $和$\frac{dy}{dx}=F'\left(x\right) $,则对于不同的微分形式,理应有$dy=f'(t)dt$和$dy=F'(x)dx$,我们的问题是,其是否能形成关联,即是否有
\begin{equation}
    dy=f'(t)dt=F'(x)dx
\end{equation}这样式子成立呢?答案是成立的。

想要证明这点,从计算着手的过程略有繁杂,我们从$dy=F'(x)dx$出发尝试导出$dy=f'(t)dt$。利用一下对复合函数求导的链式法则$\frac{dy}{dx}=\frac{dy}{dt}\cdot \frac{dt}{dx}$,或者用大家更为熟知而更简洁的$f'(x)=\frac{f'\left( t \right)}{g'\left( x \right)}$,我们引入变量$t$的构造方式同样作用于此处:
\begin{equation}
    dy=F'(x)dx=F'(x)\frac{dx}{dt}\cdot dt
\end{equation}
对于$x$的导数同样有$\frac{dx}{dt}=g'(t)$,所以继续改写原式为$dy=F'(x)*g'(t)dt$,最后由链式法则不难导出$F'(x)\cdot g'(t)=f'(t)$,于是得到了$dy=f'(t)dt$,得证$dy=f'(t)dt=F'(x)dx$。这个过程的运算和符号比较繁杂,但是原理实际上是非常清晰的,从$t\to y$的变化率由$t\to x$和$x\to y$的共同作用所构成是很显然的事情。

\lishu 例题(多个函数之间的求导和微分关系):我们有三个函数$a=a(x)$,$b=b(x)$和$r=r(x)$,我们利用微分求$a$随$r$的变化率$$
\frac{da}{dr}=\frac{a'\left( x \right) dx}{r'\left( x \right) dx}=\frac{a'\left( x \right)}{r'\left( x \right)}
$$同理,对其他函数也有$$
\frac{db}{dr}=\frac{b'\left( x \right) dx}{r'\left( x \right) dx}=\frac{b'\left( x \right)}{r'\left( x \right)}
$$ $$
\frac{da}{db}=\frac{a'\left( x \right) dx}{b'\left( x \right) dx}=\frac{a'\left( x \right)}{b'\left( x \right)}
$$

例题(隐函数的求导):暂略。

\songti 对于一阶导而言,我们可以写出$dy=f'(t)dt=F'(x)dx$,我们将这种性质叫做\underline{形式\\不变性}。它的名字很有趣,不论是形式这个词,还是不变性这个词,都是非常抽象的,但是放在这里,却又有一种天然的美感,让我们觉得如此命名是非常合理。微分的记号由莱布尼兹发明,它看起来很优美,当然,如果不出意外的话,会破坏这种美感的意外就要来了。

\lishu 例题(二阶微分是否具有形式不变性?):我们依然承袭之前的记号,设$y=f(t)$和$x=g(t)$。对于一阶导,我们有$\frac{dy}{dx}=\frac{dy}{dt}\cdot \frac{dt}{dx}=\frac{f'\left( t \right)}{g'\left( t \right)}$,继续令变量$y$对变量$x$求导,并进行一些复杂的运算:
\begin{align*}
    \frac{d}{dx}\left( \frac{dy}{dx} \right) &= \frac{d}{dt}\left( \frac{f'\left( t \right)}{g'\left( t \right)} \right) \cdot \frac{dt}{dx}
    \\&= \frac{f''\left( t \right) g'\left( t \right) -f'\left( t \right) g''\left( t \right)}{\left[ g'\left( t \right) \right] ^2}\cdot \frac{1}{g'\left( t \right)}
    \\&= \frac{f''\left( t \right)}{\left[ g'\left( t \right) \right] ^2}-\frac{f'\left( t \right) g''\left( t \right)}{\left[ g'\left( t \right) \right] ^3}
\end{align*}

这是利用导数定义求得结果的常规方法,虽然计算过程枯燥复杂,但是必然不会有错。因为与“常规方法”相对的,有一种虽然看似简单,但实际上是错误的想法:二阶微分。既然我们有$d^2f=f''\left( x \right) dx^2$,好像我们可以直接将其代入求导的分式,然后写出
$$\frac{d^2y}{dx^2}=\frac{f''\left( t \right) dt^2}{\left[ g'\left( t \right) dt \right] ^2}=\frac{f''\left( t \right)}{\left[ g'\left( t \right) \right] ^2}$$

有了正确的作为对比,我们当然知道这是错误的。在正确的推导过程中,最后一行等号后面我特意进行了化简,方便与下面的错误结果进行比较。他们都有$\frac{f''\left( t \right)}{\left[ g'\left( t \right) \right] ^2}$这一项,但是正确的答案后面还有一项减号:$\frac{f'\left( t \right) g''\left( t \right)}{\left[ g''\left( t \right) \right] ^3}$如果读者尝试用导数的定义推导过求导的商法则,则可以注意到缺少的那一部分的来源,如同积法则一般,是为加上分子对于分母额外的影响(简单来说,这里看作加法,只不过结果是负的,所以是减去)。分母的$g'^2$可以看作是为满足齐次的一种构造。显然用微分来计算的话,结果上看是缺少了计算这一部分的过程,所以错误。

\songti 对这道题的解读确实可以到此为止,因为我们的确解释明白了正确答案和错误答案的正确和错误。但是我想对于各位来说,这种浅尝辄止的回答并不能让人尽兴。错误的角度固然可以从结果的形式分析,但是更深层次的来源究竟是什么呢?

追问的理由,是因为如果我们认可从前把微分当做独立的无穷小量,等式试做无穷小量的关联来看的话,理想中的话,这种计算理应不会出现差错,微分和求导的两种计算方式结果理应殊途同归。即便用微分直接运算,直觉来讲,在推导过程中也许也可能存在一种“修正的步骤”,可以引入或者构造一些东西,使微分趋于正确答案。这种修正的步骤其实在一阶微分的推导过程中出现过,它也并非在最开始对x和t分别微分的时候,结构和形式就是一模一样的,而是引入了一个利用链式法则换元的过程,最终构造了$dy=F'(x)*g'(t)dt=f'(t)$。

但是对于二阶微分,这仿佛并不仅仅是形式上的暂时不完美,而是一种错误,而且错误又是非常彻底的,毫无修正的办法和可能。这令我们十分好奇错误的原因。

首先我们会去思考记号问题,也就是$d^2y$是否合理?为了想清楚这点,我们又难免追问下去:$dy$的记法是否合理?思考二阶微分的错误或许有点难,但是先从本就正确的一阶微分开始思考,或许能帮我们找到突破口。$dy=f'(t)dt=F'(x)dx$的形式不变性作为结果的体现,导出了一阶求导和微分之间的关系。那么对于二阶微分,我们可以先尝试写出$d^2y=f''(t)dt^2=F''(x)dx^2$,然后发现它是错的(验算留做习题)。

为什么对于一阶微分,它是正确的呢?如果我们细品我们推导它的过程中采取的手段,首当其冲的是链式法则,但是链式法则理应同样可以应用在二阶导数,所以它并不是导致二阶微分和一阶微分出现差异的地方。事实上,在我们关注$F\left( x \right) =F\left[ g\left( t \right) \right]$的时候,一定能注意到还有一个$dy$作为它们统一的结果。无论通过什么渠道,所计算的dy都是一致的,这源于$y=f(t)=F(x)$,而能如此写下这个连等式,又离不开两个变量$x=g(t)$的自然约束。一阶导直接反应了同一个作为f和F两种映射法则的结果的y的变化,虽然两种映射f和g对应的两种导数$f'$和$g'$依然不同,但是如果仔细观察$f'$和$g'$的求导过程,其来源都是对y变化的反馈。

而二阶导,就没有一个y作为统一结果了。我们要各自分析$f'$和$F'$的变化率,且理应注意到在自然约束$x=g(t)$恒成立的情况下,$f(x)$和$f(t)$的结果是完全不一样的,$F(t)$也肯定不能等同于是$F(x)$。一定要注意到书写为函数的$F(x)$是作为一个映射过程,而不是结果(结果记作了y)。f和F是不同的映射路径,t和x是不同的映射的出发点。因为其共同导向了y,在原函数中,可以通过代入g变换,使两个映射互相转换,f变成F,或者F变成f。但在一阶导处,已经不能有类似的转换了,让f'→g'的操作并不是直接带入$x=h(t)$,这个过程已经可以说明导函数并非可以通过自然约束相关联,我们考虑的二阶导,是作为一阶导的一阶导,作为映射的变化率。

所以我们要关注从关注变量的性质,转换到关注求导映射的性质上来。

由于基于不同的变量x和t的函数,求导的映射过程是截然不同的,就如同现实中某一个物理量有与另外两个物理量的直接管理约束,但从变化的角度,基于另外两个物理量的变化关系是截然不同的。把不同的变量看成一个有方向性的东西,这种方向是一种变量到变量之间函数映射的方向,这样一个函数对应一个映射的路径,我们考虑对于同样的出发点x和t,两种不同的路径所体现的两种不同的有关变化率的性质,会为我们导向不同的结果。这个时候就无关于y层面的约束了,因为对于$?=f(t)$,$f'$只取决于函数f而不会有所改变,而对于$?=F(x)$,$F'$只取决于函数F而不会有所改变。

我们将其称作求导对函数路径的依赖,我们必须时时刻刻关注我们对什么路径进行求导。

那么到二阶微分,$\frac{dy}{dt}$和$\frac{dy}{dx}$的性质上的不同,注定了$d\left( \frac{dy}{dx} \right) $和$d\left( \frac{dy}{dt} \right) $的不同,所以对于$d^2y$没有所谓的形式不变性,只能作为“求导两次”的记号,对单一路径使用。对于不同的路径,$d^y$的意义是完全不一样的。













在初学导数时,我想我们所接触的例题应该都不约而同的是求$f(x)=x^2$的导数。在最初,我们不把$\Delta x$当成0而放在分母,并在最后导出了一个$2x+\Delta x$的极限,并在此终于把$\Delta x$当成了0而忽略。试问这里的忽略,是认为它存在,只是太小而不去计算呢,还是认为它明明就不存在,所以忽略的很正常呢?注意到,求导中真正使我们导出2x的原因,只在于完全平方公式的结构,这又正是二次函数$f(x)=x^2$的性质的体现,和我们所设的Δx的确无关。就算不让Δx→0,我们依然可以顺利导出2x的部分。从这里我们可以完全肯定,Δx的的确确是一种虚设的量,其意义单纯是为了导出函数决定变化量的性质的部分,而不在于增量本身。

借用物理学中的一些概念,我更习惯把趋于0的dx称为虚位移,随着$dx\to0$而趋于0的dy称为虚功。这种叫法或许更能有效地反应出其真正的内涵。我们引入无穷小的变化量,并非真正用于计算函数,而是用于导出函数的变化性。之所以用“性”这个词,意在更注重其作为一个函数内在蕴含的性质,而不受我们外在所虚设的dx所影响。









有很多人在解释这些事情的时候会说,求导用的记号是分式,但实际上他只是一个记号,只是在一阶微分的时候恰好符合了分式的运算法则。我是很不喜欢这种说辞的,因为从来源上看,一阶的微分恰恰是在我们考虑函数的性质,并由分式表达速率的性质出发,于是自然而然得到的记号法则,所以用“恰好”这个词来解释不变性的理念是一种曲解。我们要是对其追问为什么要使用分式作为记号,亦或者追问这里的分式是否是除法,他们的回答必然会含糊不清,悄悄携带一种偏否定而又不全否定的态度敷衍了事。

我们其实完全可以肯定地说,这里用分式和d的记号并不荒谬,分式就是除法,微分就是无穷小量。反而求导这一理念,$\frac{d}{dx}$的的确确是抽象的记号,是用来表示函数到函数的映射的,由微分衍生出来的高度抽象化的事情。

当然,对于导数的定义而言,除法的确就是除法,是人类研究如何度量速率所找到的方法。但是对于不考虑人类用以描述自然科学的手段,只关注事物本身,它的速率的确应是它内在具有的一个性质。无关于差分还是微商,都是我们用以导出速率这一性质的手段。客观存在的性质都是的确客观存在的。这个时候再去考虑用映射的角度处理函数,进行求导,抽象出性质读取的本质,抛去那些容易产生误解的取极限的手段和过程,不失为一种更遵循客观自然的理解角度。


求导是一个函数向另一个函数的映射,例如由$f(x)=x^2$到$f'(x)=2x$的映射。对于求导符号$\frac{df}{dx}$来说,其中的$\frac{d}{dx}$可视作一个整体,这个过程可以将其看作是$\frac{d}{dx}f\left( x \right) \rightarrow f'\left( x \right) $。求导是原函数向导函数的映射,一种把利用形式逻辑将极限运算简单化的操作。而微分往往更重视导数意义的本身,也就是无穷小量变化的比值。我们会把$\frac{df}{dx}$视作$df$和$dx$的比值,并且可以拆开,写作$$df=f'\left( x \right) dx$$对于二阶导数而言,我们的求导记号是$\frac{d^2f}{dx^2}=f''\left( x \right) $,所以微分的写法可以写作$d^2f=f''\left( x \right) dx^2$,对于高阶导数,我们依此类推。

\end{document}