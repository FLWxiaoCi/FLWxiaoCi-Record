\documentclass[fontset=none]{ctexart}
\ctexset{fontset=windows}   %指定字体库为windows
\usepackage{fancyhdr}
\setlength{\headheight}{13pt}  % 调整页眉高度
\usepackage{float}
\usepackage{amsmath,amsthm,amssymb,amsfonts}
\usepackage[left=3cm,right=3cm,top=2.5cm,bottom=2.5cm]{geometry}
\usepackage{graphicx}
\usepackage{epstopdf}
\usepackage{subfigure}
\pagestyle{fancy}
\fancyhf{} % 清空默认页眉页脚
\fancyhead[L]{QQ:17950026}
\fancyhead[R]{作者:姜鱼}
\fancyhead[C]{兼听则明,偏听则暗}

\usepackage{lmodern}
\makeatletter
\renewcommand{\maketitle}{
  \begin{flushleft} % 整体左对齐
    {\huge\bfseries \@title \par} % 标题
    \vspace{0.5em}
    {\large \bfseries \textit{作者:} \@author \par} % 作者(添加自定义文字)
    \vspace{0.5em}
    {\large\bfseries \textit{日期:}\@date} % 日期(调整格式)
  \end{flushleft}
  \thispagestyle{empty} % 可选:隐藏页眉页脚
  \vspace{1cm} % 标题块后间距
}
\makeatother

\begin{document}
\title{期望的定义}
\author{姜鱼}
\date{\today}
\maketitle
\songti

对于离散型随机变量$$E\left[ X \right] =\sum_x^{}{xP\left\{ X=x \right\}}$$

我们定义它的期望,类似于均值,是这组数据中能得到的平均结果。但是不同于直接取均值的理念,我们从期望的角度,用性质为其定义:如果一处变量在整体中的占比(或者说概率)越大,那么期望就更有可能落在这个数上;此外,概率相同的多个变量在,我们再考虑从变量本身的大小出发,取其均值,定义期望落在这个均值上。

基于这样的思路,我们最终定义了以$x × P$为单元再对其求和的计算方法。实际上可以理解为加权,以\underline{变量的数值}加权\underline{变量的占比}。

离散变量的概率,是对每处的变量值考察其概率。概率学依赖于集合,所以我们永远可以视其为某一部分集合$A$在另一部分集合$B$中的占比,个体之于整体。所以过渡到连续变量,则是对一段变量出现的区间考察其概率,相当于范围之于整体。

所以概率需要积分:$$P\left\{ a\leqslant x\leqslant b \right\} =\int_a^b{p\left( x \right) dx}$$

所以需要用一段区间中每处变量为其概率的加权:$$E\left[ x \right] =\int_a^b{xp\left( x \right) dx}$$

当然,乘法是有交换律的,我们也可以反向思考由概率为变量加权进行求和或者积分,殊途同归。

\end{document}